
\chapter{Introduction}\label{sec:introduction}

A lot of research has been done in the field of Unmanned Aerial Vehicles (UAV) in the last years leading to a strong improvement in planning \cite{he} as well as in control [\cite{colling}, \cite{hehn}].  Another research field is machine learning \cite{lup} which is suitable to enhance the performance of aerobatic maneuvers but seams to have a downside regarding motion planning and trajectory generation in dense environments. \newline

Speaking of trajectory planning, there are two different strategies which are pursued. On the one hand, the geometric and the temporal planning are decoupled  \cite{bou} on the other hand, geometric and temporal information are coupled and the trajectory is the result of a minimization problem. For the couplet problem one can make use of the differential flatness of a quadrocopter to derive constraint on the trajectory. Then formulate a cost-function which could be the trajectory-time \cite{hehn} or the total snap \cite{mellinger} (second derivation of acceleration).



