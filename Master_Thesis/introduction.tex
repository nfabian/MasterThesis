
\chapter{Introduction}\label{sec:introduction}

\section{State of the Art}\label{sec:state}

A lot of research has been done in the field of Unmanned Aerial Vehicles (UAV) in the last years leading to a strong improvement in planning \cite{he} as well as in control [\cite{colling}, \cite{hehn}].  Another research field is machine learning \cite{lup} which is suitable to enhance the performance of aerobatic maneuvers but seams to have a downside regarding motion planning and trajectory generation in dense environments. \newline

Speaking of trajectory planning, there are two different strategies which are pursued. On the one hand, the geometric and the temporal planning are decoupled  \cite{bou} on the other hand, geometric and temporal information are coupled and the trajectory is the result of a minimization problem. For the couplet problem one can make use of the differential flatness of a quadrocopter to derive constraint on the trajectory. Then formulate a cost-function which could be the trajectory-time \cite{hehn} or the total snap \cite{mellinger} (second derivation of acceleration). \newline

Another aspect of planning is exploring the state space in the first place. A strong tool to do so are incremental search techniques as for instance the A* \cite{lik} or the RRT* algorithm \cite{richter}. The sampling points of the solution of the incremental search can then be used as the vertices for the polynomial optimization.

\section{Quadratic Programming}\label{sec:quadratic}

\subsection{Constrained Quadratic Programming}

Quadratic Programming (QP) is a special case of optimization problem in which a quadratic function is optimized with respect to its optimizations variables (which are represented with the vector $x$ in Equation \ref{equ:quadratic})

\begin{equation}
 f(x)  = \frac{1}{2} \cdot x^T Q x + c^T x 
\label{equ:quadratic}
\end{equation}

The optimization is performed under linear constraints on the optimizations variables. Whereas a distinction between equality($ E\mathbf{x} = \mathbf d $) and inequality constraints ($ A\mathbf{x} \leq \mathbf b $) has to be made. 

In case there are only equality constrains, the solution to the QP is given by the linear system in Equation \ref{equ:equality} :


\begin{equation}
\begin{bmatrix}
   Q & E^T \\
   E & 0
\end{bmatrix} 
\cdot
\begin{bmatrix}
   \mathbf x \\
   \lambda
\end{bmatrix}
= 
\begin{bmatrix}
   -\mathbf c \\
   \mathbf d
\end{bmatrix}
\label{equ:equality}
\end{equation}


%where $\lambda$ is a set of Lagrange multipliers.

\subsection{Unconstrained Quadratic Programming}
The constrained QP gets ill-conditioned for a large amount of segments or for high order polynomials (which both lead to large and sometimes sparse matrices). 
To reduce the number of optimization variables, and therefore the size of the matrices, the constrained QP can be converted into a unconstrained QP. \newline
In other words, the polynomial coefficients are no longer the optimization variables but the free endpoint derivatives are optimized. The exact formulas of the unconstrained QP can be seen in Section \ref{sec:polynomialQP}. Whereby the linear term $x^T z$ from the quadratic cost-function in Equation \ref{equ:quadratic} is equal to zero for the polynomial optimization and therefore absent in Equation \ref{equ:uncon_cost}.  \newline









