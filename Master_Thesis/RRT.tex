\chapter{RRT}\label{chap:RRT}

\section{General}

The goal of this Thesis was not only to generate a numerical stable, snap optimized polynomial trajectory but also to explore a dense (indoor) environment and plan an aggressive trajectory in between the obstacles. Hence, the Rapidly-Exploring Random Tree (RRT) algorithm is used to find a collision free straight-line solution through dense environments. The sampling points oft the RRT (or RRT*) algorithm are then used as the vertices in the polynomial optimization.

\section{Algorithm}

\subsection{RRT}

RRT is a computational efficient algorithm to find a path in a high dimensional space by randomly building a space-filling tree. The sampling points are drawn randomly from the sample space and the tree grows incrementally. 
For each new sample the algorithm attempts to build a collision-free connection to the nearest state in the tree. If a collision-free connection is possible the sample and the connection are added to the tree. \newline

The RRT algorithm can depicted schematically:


\begin{enumerate}
  \item Sample
  \item Find nearest state in tree
  \item Try to build a collision-free connection to the nearest state
  \item If feasible, add the sampled state and the connection to the tree
\end{enumerate}

\subsection{RRT*}

In contrast to the RRT algorithm the RRT* (or RRT Star) algorithm not only tries to connect to the nearest state in the tree but to several states near the sampled state.



\begin{figure}[h]
   \centering
   \includegraphics[width=1\textwidth]{pics/initialSolution.png}
   \caption{Ein Bild.}
\end{figure}


\begin{figure}[h]
   \centering
   \includegraphics[width=1\textwidth]{pics/Vertex_in_middle_2.png}
   \caption{Ein Bild.}
\end{figure}

\begin{figure}[h]
   \centering
   \includegraphics[width=1\textwidth]{pics/section.png}
   \caption{Ein Bild.}
\end{figure}



%\begin{figure}[h]
%   \centering
%   \includegraphics[width=1\textwidth]{pics/section_and_time.png}
%   \caption{Ein Bild.}
%\end{figure}


\begin{figure}[h]
   \centering
   \includegraphics[width=1\textwidth]{pics/Nlopt_after_sectionAndTime.png}
   \caption{Ein Bild.}
\end{figure}