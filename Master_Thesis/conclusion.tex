\chapter{Conclusion}\label{chap:Conclusion}

The goal of this master thesis was to explore a densely packed environment and then plan a collision-free trajectory for a dynamic UAV flight. \newline

To explore the environment, the RRT* algorithm was applied. The parameters of RRT* have been tuned to serve the subsequent nonlinear optimization in an optimal manner. The vertices of the straight line solution have then been used to generate a polynomial trajectory. Here, the numerical advantages of the unconstrained Quadratic Programming (QP) take full effect. The unconstrained QP has been tested for a large number of optimization variables and is numerical stable for trajectories with more than 2000 segments! \newline

Furthermore, the nonlinear optimization benefits from the numerical stability of the unconstrained QP. The reduction of the optimization variables (i.e. excluding the endpoint derivatives $d_P$ from the cost function) led to a even better performance. The nonlinear optimization has been successfully tested for a trajectory with 200 segments. The NLopt ending criteria was $J_{rel} = 0.5$ and the algorithm converged. \newline

Flight tests in the ETH Vicon room have shown that the path planning algorithm is suitable for online planning. The UAV flew on a smooth, snap minimized trajectory from the current position between multiple obstacles to the user-specified goal position. The current position, which is also used by the controller, is provided by the Vicon motion tracking system. 




